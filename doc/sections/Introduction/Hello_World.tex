% !TeX root ../../tgame_doc
% !TeX spellcheck = en_GB

Before presenting the minimal Hello World program let's start with a simpler goal: starting a game that does nothing, which is arguably the most minimal program one could think of in this context. One way to do it is through the following code

\begin{figure}[htb]
\begin{snippet}
import tgame as tg

def init(ctxt):
	ctxt.room_new("Dummy")
	
tg.wrapper(init)
\end{snippet}
\end{figure}

The first line of the code is simply telling python to import the \code{tgame} module in our program. Next we define the function \code{init} which takes only one argument, a \code{context} object and create the first room called \textit{Dummy}. Finally we pass \code{init} to the wrapper which uses this function to initialise the game. When this script is executed the \textit{Dummy} room will be displayed, i.e.\ an empty room.
In order to actually print a message we need to define an object whose purpose is to display it and then instantiating it in our room. This is done in the following.

\begin{figure}[htb]
\begin{snippet}
import tgame as tg

@tg.active_object
class hello_world(object):
	def ev_draw(self):
		self.s.addstr(0, 0, "Hello World")

def init(ctxt):
	ctxt.room_new("Dummy")
	
	hello_world(ctxt)
	
tg.wrapper(init)
\end{snippet}
\end{figure}

The first new thing to introduce is the \code{hello\_world} class, decorated with \code{@tg.active\_object}, which makes it an object that execute code when events are triggered and provide us with handler to manage it. One of this event is the \code{ev\_draw}, executed once per step, and to specify an action to perform we simply define the homonymous method. In this case we access the \code{curses.screen} that is stored in \code{self.s} by the decorator and then execute the \code{addstr} method to print a string "Hello World" on position $(0, 0)$.\\
The second thing we add is the instantiation of this object in the \textit{Dummy} room. To do so we simply call the constructor passing the context as first argument.

