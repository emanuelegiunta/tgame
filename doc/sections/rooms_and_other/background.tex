% !TeX root ../../tgame_doc
% !TeX spellcheck = en_GB

\paragraph{Methods}
Background object are the first one to be drawn on screen after a reset, usually before the draw event is performed. 
This is the reason why printing anything on screen before the draw event will not produce any effects.
Backgrounds are currently meant to be mainly loaded once and forgotten, however basic manipulation through getters and setters is provided

\medskip
\func{background}{context, bkg \keyword{None}}{
	Create a background object using information on \textit{bkg}. 
	If \textit{bkg} is \code{None} the background is set to empty. 
	If \textit{bkg} is a string the constructor interpret it as a path and try to load a \code{.bkg} file - which should be of the same size of the current main screen.
	If \textit{bkg} behave as a dictionary, the constructor extract the background from it.
}

\func{ch\_set}{y, x, c, a \keyword{curses.A\_NORMAL}}{
	Set the character in position (\textit{x}, \textit{y}) to be \textit{c} with attribute \textit{a}.
	If the position is not in the screen an error is raised \EG{which one?}
}

\func{ch\_get}{y, x}{
	Get the character in position (\textit{x}, \textit{y}) and returns a tuple (\textit{c}, \textit{a}) containing the character in the given position and the attribute.
	If the position is not inside the screen an error is raised.
}

\func{ch\_del}{y, x}{
	Remove the character in position (\textit{x}, \textit{y}).
	If the position is not inside the screen an error is raised.
	\remark{running \code{ch\_del(y, x)} should be preferred over the semantically equivalent \code{ch\_set(y, x, " ")}.}
}

\func{draw}{ }{
	Draw the background on screen
}

\paragraph{File format} \EG{Todo}