% !TeX root ../../tgame_doc
% !TeX spellcheck = en_GB

\paragraph{Methods and variables}
The decorator \code{@active\_objects} is the most essential. Its main goal is to equip the given class with basic methods that allows the instances to interact with the context \EG{ref} and perform basic actions such as activate, deactivate and destroy themselves. 
Crucially, the \code{context} object assumes that all the instances he receives are at least decorated as active objects, for instance by checking if they are inactive before executing an event.\\
Below a list of methods and variables added/modified by the decorator

\medskip
\vari{.active}{Read Only}{
	Boolean values modified by the methods \code{activate} and \code{deactivate}. If \code{True} for a given instance in the current room, the context will call its event methods in the main loop.
}

\vari{.\_c}{Private, Read Only}{
	Context object \EG{ref?}.
}

\vari{.\_s}{Private, Read Only}{
	Associated screen object.
}


\func{\_\_init\_\_}{context, *argv, screen \keyword{None}, **kwarg}{
	Attach the given instance to \textit{context} and bind the screen. If the decorated class implements \code{\_\_init\_\_} this is executed with \textit{argv, kwarg}; otherwise these values are passed to \code{ev\_create} if implemented. If both \code{\_\_init\_\_} and \code{ev\_create} are defined, only the first method is invoked and a \code{RuntimeWarning} is raised.\\
	\remark{observe that the new \code{\_\_init\_\_} has two extra argument with respect to the previous one/the creation event.}
}

\func{activate}{ }{
	Set \code{active} to \code{True}.
}

\func{deactivate}{ }{
	Set \code{active} to \code{False}.
}

\func{destroy}{  }{
	Destroy the object and perform the \code{ev\_destroy} event.
}

\paragraph{Examples} \EG{TODO}