% !TeX root ../../tgame_doc
% !TeX spellcheck = en_GB

\paragraph{Methods and variables}
The decorator \code{@active\_object} is the most essential. Its main goal is to equip the given class with basic methods that allows the instances to interact with the context \EG{ref} and perform basic actions such as activate, deactivate and destroy themselves. 
Crucially, the \code{context} object assumes that all the instances he receives are at least decorated as active objects. This is essential for instance when checking whether they are inactive before executing an event.\\
Below a list of methods and variables added/modified by the decorator is provided.

\medskip
\vari{.visible}{Read \& Write}{
	Boolean value \code{True} by default. If \code{False} the draw event, see \code{ev\_draw} \EG{ref}, is not performed.\\
	\remark{All the other events happen as usual and in particular an invisible active object is still able to draw on screen if drawing functions are called outside \code{ev\_draw}, for instance in \code{ev\_step\_end}}
}

\vari{.permanent}{Read \& Write}{
	Boolean value \code{True} by default. If \code{True} the instance will only be preset the context's room at the moment of instantiation. Otherwise the instance will be carried in the next room while performing a room transition.
}

\vari{.layer}{Read \& Write}{
	Integer value indicating the layer in which the current object is drawn.\\
	Modifying this value will produce visible effects from the next step.
}

\vari{.active}{Read Only}{
	Boolean value modified by the methods \code{activate} and \code{deactivate}. If \code{True} for a given instance in the current room, the context will call its event methods in the main loop.
}

\vari{.c}{Read Only}{
	Associate context object \EG{ref?}.
}

\vari{.s}{Read Only}{
	Associated screen object. Setted at creation time through the flag \textit{s} or \textit{screen}, both accepting a \code{curses.window} object
}

\func{\_\_init\_\_}{context, *argv, **kwarg}{
	Attach the given instance to \textit{context}. 
	If \textit{screen} a keyword argument in \textit{kwarg} then it set the instance view to \textit{screen}, otherwise the \textit{context} screen is uses.\\
	If the decorated class implements \code{\_\_init\_\_} this is executed with \textit{argv, kwarg}; otherwise these values are passed to \code{ev\_create} if implemented. If both \code{\_\_init\_\_} and \code{ev\_create} are defined, only the first method is invoked and a \code{RuntimeWarning} is raised.\\
	\remark{observe that the new \code{\_\_init\_\_} has one extra argument with respect to the previous one/the creation event.}
}

\func{activate}{}{
	Set \code{active} to \code{True} and activate the current instance.
}

\func{deactivate}{}{
	Set \code{active} to \code{False} and deactivate the current instance.
}

\func{destroy}{}{
	Destroy the object and perform the \code{ev\_destroy} event.
}